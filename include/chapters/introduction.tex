\section{Введение}
Прецизионные усилители (ПУ) отличаются стабильным коэффициентом усиления в заданном диапазоне рабочих частот и температур.
Они широко используются в системах сбора данных, устройствах выборки и хранения сигналов, телеметрических системах и измерительной технике.

Источниками сигналов для ПУ обычно являются различного рода датчики.
Сигналы, поступающие в ПУ, характеризуются частотным спектром и динамическим диапазоном.

Основным звеном прецизионного усилителя является операционный усилитель (ОУ).

\subsection{Анализ параметров прецизионного усилителя}
Коэффициент усиления $K_0$ является одним из наиболее важных параметров ПУ.
Значение $K_0 >> 1$ указывает на возможность усиления слабых сигналов, стабильность $K_0$ позволяет получить достоверную информацию о величине входного сигнала.
$K_0 = 1000$ позволяет легко реализовать масштабирование сигнала.

В качестве элементной базы целесообразно выбрать операционный усилитель.

Для стабилизации ($\delta K_0 \le 2 \%$) при проектировании ПУ на базе операционных усилителей используют резистивную ООС.
ОУ включается по схеме НУ или ИУ.

Наличие нижней рабочей частоты допускает применение разделительных конденсаторов на входе и выходе, а при необходимости и между отдельными каскадами.
Балансировку нуля ОУ исключаем.
Наличие верхней частоты необходимо при выборе ОУ по быстродействию и по скорости нарастания.

Значение коэффициента частотных искажений на нижних частотах позволяет рассчитать номинальные значения разделительных конденсаторов.
При этом $M_{н1} = \frac{M_н}{n_C}$, где $n_C$ --- число конденсаторов.

По максимальной амплитуде входного сигнала судят о пределах изменения выходного сигнала усилителя.
При относительно высоком сопротивлении $R_вх > 1 \text{ кОм}$ целесообразно использовать НУ, так как он обладает большим входным сопротивлением.

